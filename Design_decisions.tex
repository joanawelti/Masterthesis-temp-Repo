\section{Design Decisions}
%When building a prototype of a system, there is the option to either build an evolutionary or a throwaway prototype. Evolutionary prototypes are eventually evolved into the the full system\cite{interactive_systems} and code is reused. Throwaway prototypes are discarded and the final system is rewritten from scratch. 

\subsection{Framework options}\label{sec:framework_options}
For the implementation of the PowerPedia prototype, Recess is the framework that was chosen. First of all, it was selected because of its native support for REST. Second, the Energy Server on the eMeter gateway used the same technology, which resulted in synergies of which could be taken advantage. Thus, there was already some experience with the framework and it proved to be a good choice for the system.

At the current time, the situation looks slightly different. Recess is still used partly on the Energy Server, but some problems have emerged. Recess v0.20 was publicly released in August 2009. For the last two years, the developers of the project have not been very active\footnote{See comments on the plans for 2010: \url{the http://www.recessframework.org/page/new-recess-framework-website}}. The main issue with Recess is that it requires PHP version 5.2.3+, but it doesn't work with PHP 5.3 yet. PHP version 5.2 is not supported anymore by the PHP team\footnote{"All PHP users should note that the PHP 5.2 series is NOT supported anymore. All users are strongly encouraged to upgrade to PHP 5.3.7." \url{http://ch.php.net/archive/2011.php}}. In order to stick to Recess, an outdated version of PHP has to be used, which, of course, is not desirable.  

Mainly due to this problem with the Recess framework, I decided to rewrite PowerPedia from scratch. This includes both the signature server and the harvester in order to be able to get rid of the framework. 
Wikis and Facebook applications were reconsidered, including the reasons why they were not favored in the first place. Other options that were evaluated are Ruby on Rails, Django, Grails, Google Web Toolkit (GWT), and Struts as a representative for the Java EE applications family.    

The possible frameworks were evaluated according to following criterions:
\begin{itemize}
 \item \textbf{Support for REST architecture} Communication between the mobile phone clients and the eMeter gatewy is implemented as HTTP calls following a RESTful architecture. By applying the same principle to the PowerPedia prototype, no additional communication libraries had to be installed. In order not to have to change the existing mobile phone clients too much, the new PowerPedia version should be RESTful as well.
 \item \textbf{MVC pattern} In order to build clean, easy to maintain web applications, the MVC pattern is basically the standard technique to use. Support for MVC is desirable for a new framework. 
 \item \textbf{User accounts} Privacy concerns when publishing electricity data have to be considered. User accounts are needed, which should be supported by the framework.
 \item \textbf{Collaboration} Collaboration is an important aspect of the new version of PowerPedia. The framework should facilitate building a so-called Web 2.0 application.
 \item \textbf{Internationalization} Multiple languages (at least English and German) need to be supported.
 \item \textbf{Learning curve} Because of the time constraints of the project, the framework must be easy to learn.  
 \item \textbf{Usage in similar Projects} Are there any other collaborative platforms that use this framework?
\end{itemize}

\subsubsection{Wiki}
A wiki is an online collaborative writing tool, a web space where anyone can add content and edit articles that have already been published\cite{wikis_collaboration}. 
%Wikis have become quite popular over the last few years. They are used in enterprises to have a central knowledge base as well as by individuals to collect information on a topic in a collaborative way.  
They are used for a wide variety of topics, such as travel~\footnote{Travel Wiki \url{http://wikitravel.org/}} or health~\footnote{Health Wiki \url{http://www.ganfyd.org/}}. The most prominent example is probably Wikipedia, which has a very large user base~\footnote{WikiStats by S23, S23Wiki, 2008-04-03, retrieved 2007-04-07, \url{http://s23.org/wikistats}} and around 90'000 active contributors.

Because of the popularity of Wikis, users are familiar with the concept. Wikis encourage collaboration by making the editing and creation of pages easy and available for all users. They are various implementations in languages such as PHP or Java. Generally, wikis offer rich functionality such as full text search, user accounts and version control. Different languages are supported by having different wiki versions for each language.  

%The structure and format of wiki pages are specified with a simplified markup language in most implementations, although the syntax can vary greatly among them. Especially implementations that don't use "WYSIWYG" ("What You See Is What You Get") editing might be difficult to use for novel users as they first have to become familiar with the specific markup language\footnote{Wikipedia conducted a usability study in 2009: "Every user in this study struggled to get a basic grasp of the editing interface." \url{http://usability.wikimedia.org/wiki/UX_and_Usability_Study}}\cite{wikis_collaboration}.
The structure and format of wiki pages are specified with a simplified markup language in most implementations. Wikis are good for creating simple content. Complex content such as tables or figures can be challenging\cite{wikis_collaboration}.

Because wikis are software that can be used as they are, generally there is no built in REST or MVC support. Some of the available wiki software is open source, which means that they could be extended and customized3.

\subsubsection{Facebook application}
Currently, Facebook has over 800 million active users\footnote{Facebook statistics \url{http://www.facebook.com/press/info.php?statistics}}. It offers an API for programmers to build applications that can integrate with many aspects of Facebook such as news feeds and notifications. User management is completely managed by Facebook. Applications are loaded into Facebook via iframes, but have to be hosted externally.

Energy Monsters\footnote{Energy Monsters Facebook application \url{http://www.facebook.com/apps/application.php?id=102704939189}} is such an application. It was developed by the same team that created the Velix project (see section ~\ref{sec:household_level_feedback}). Analog to Velix, users are encouraged to regularly publish their electricity meter readings. %According to the application description, the "Energy Monsters application helps you to track your energy consumption. Insert the kWh displayed on your electricity meter once a week and discover your energy monsters. You will get feedback on your average consumption, get tips on how to reduce your consumption and compare yourself to other users. This application will constantly be extended by new functions." 
Unfortunately, it did not receive much attention. 	

Facebook applications can be a good way to supplement a web platform. It is not the intention of Facebook applications however to be used as the main platform for a complex system.

\subsubsection{Dynamic language web application frameworks}
\minisec{Ruby on Rails}
Ruby on Rails~\footnote{\url{http://rubyonrails.org/}} (RoR) was first introduced in 2003 as a open-source web framework for the Ruby programming language. It consequently implements the Model-View-Controller pattern. 
RoR strives to make web application development easier by making certain assumptions, which is reflected in its \textit{Convention over Configuration} design principle. Also, it empathizes the \textit{Don't repeat yourself (DRY)} concept, which means that code that is used in multiple places should only be written once. 
Using REST as the pattern for web applications is another one of its guiding principles, which sets it apart from other MVC-frameworks. Domain objects are treated as resources and can be managed with the standard HTTP verbs. Internationalization is supported with the I18n\footnote{I18n gem description \url{http://guides.rubyonrails.org/i18n.html}} gem that is shipped with Rails.

Various collaborative web platforms have been implemented with RoR. As mentioned in~\ref{sec:collaborative_web_platforms}, Github is one of them. 43Places\footnote{43Places \url{http://www.43places.com/}} (platform to share stories about places and ask for travel advice), Writeboard\footnote{Writeboard \url{http://www.writeboard.com/}} (platform to collaboratively work on documents with other people), or Ravelry\footnote{Ravelry \url{http://www.ravelry.com/}} (community platform for knitting and crochet where projects can be shared with other people) are other examples.

There has been some controversy about the scalability and performance of RoR. For example, Twitter moved from Ruby on Rails to Scala. It is still used for the front-end, but it was not performant and reliable enough for the backend\cite{twitter_ruby_scala}. On the other hand, Scribd, another one of the largest Rails sites, has used it right from the start and continues using it\footnote{The site delivers on average 36MB of traffic every second. See \url{http://www.scribd.com/doc/49575/Scaling-Rails-Presentation-From-Scribd-Launch} and \url{http://www.scribd.com/doc/27168812/Oh-Shit-How-to-Break-a-Large-Website-and-how-not-to-PDF-version} for more details on how Scribd deals with performance/scalabilty on RoR}.
There is not built-in user management.

\minisec{Django}
Django\footnote{Django \url{https://www.djangoproject.com/}} is a high-level Python Web framework that was created to ease the development of complex, database-driven websites. Similar to Rails, it also adheres to the \textit{DRY} principle and focuses on automating as much as possible. %Data models can be written entirely in Python as the framework comes with an object-relational mapper (ORM).
It has a full support for multi-language applications and is able to automatically create an admin interface that lets authenticated users add, change and delete objects. Django is known for its good documentation, which makes it easy to learn.

In order to build RESTful APIs, mini-frameworks such as django-piston\footnote{django-piston http://pypi.python.org/pypi/django-piston} or django-rest-interface\footnote{django-rest-interface http://django-rest-framework.org/} can be included in the project. Django comes with a user authentication system, which facilitates user management. 

Collaborative web platforms powered by django include Mahalo\footnote{Mahalo \url{http://www.mahalo.com/}} (a human search engine/question and answer site, where users can pose and answer questions and leave comments) or Key Ingredient\footnote{Key Ingredient \url{http://keyingredient.com/}} (platform to collect, share and publish recipies).   

\minisec{Grails}
Grails\footnote{Grails \url{http://grails.org}} is an open source web application platform that uses the Groovy programming language, which is based on the Java platform. Because of this, it is able to build on established Java technologies such as Spring and Hibernate. This full-stack framework supports internationalization and is designed around the MVC pattern as well as the \textit{DRY} principles.
For REST support, there is a plugin called JSON-REST-API\footnote{\url{http://www.grails.org/plugin/json-rest-api}}.

Compared to traditional Java web frameworks, Grails doesn't need to be configured so heavily. Instead, it uses a set of predefined rules and conventions. Unfortunately, Groovy is not a very widespread programming language\footnote{E.g according to the TIOBE index: \url{http://www.tiobe.com/index.php/content/paperinfo/tpci/index.html} (Index for September 2011)}. On the other hand, because of Groovy and its base on the Java platform, performance and scalability are less a problem than with other frameworks\footnote{A Grails vs Rails Benchmark can be found here \url{http://www.grails.org/Grails+vs+Rails+Benchmark}}. 

There are some small collaborative websites that use Grails. Examples include ChatNear Me\footnote{ChatNear Me \url{http://www.chatnearme.com/}} (a location based real-time chat service) and and Your Divebook\footnote{Your Divebook \url{http://www.yourdivebook.com/}} (a community portal for divers). 

\subsubsection{Others}
\minisec{Google Web Toolkit}
Google Web Toolkit (GWT) is a development toolkit that enables developers to build and optimize browser-based applications\footnote{GWT \url{http://code.google.com/webtoolkit/}}.  It allows to write complex front-end applications using Java, which is then complied into highly optimized JavaScript that runs across all browsers. Instead of having to use multiple technologies such as AJAX, HTML or JavaScript, with GWT everything is Java. Many Google products such as Google Wave\footnote{Google Wave \url{wave.google.com/}} (a shared space for discussing and working together) or Orkut\footnote{Orkut \url{www.orkut.com/CommunityJoin}} (a social networking and discussion site) use this technology. GWT is intended for rich client applications. 
%and can be considered a framework for \textit{Rich Internet Applications (RIA)}. RIAs~\cite{ria} are a new generation of web applications that share many properties with desktop applications. They are rich in terms of multimedia support and user interface widgets and responsive by supporting client-side computation. 

With GWT, the server side of the application is not yet solved. For the server side, any technology can be used. 


\minisec{Struts}
Apache Struts\footnote{Apache Struts \url{http://struts.apache.org/}} is a framework for developing Java EE applications\footnote{Java EE \url{java.sun.com/j2ee/}}.
According to the Java EE documentation, the platform provides an API and runtime environment for "developing and running large-scale, multi-tiered, scalable, reliable, and secure network applications."\footnote{Your First Cup: An Introduction to the Java EE Platform \url{http://download.oracle.com/javaee/6/firstcup/doc/gkhoy.html}}.
To overcome the problem of mixing application logic with representation that arises in standard Java EE applications, Struts encourages the MVC architecture. It works well with REST applications and also supports internationalization. 

Struts "stands on the shoulders of giants"\footnote{Struts for Newbies \url{http://struts.apache.org/index.html}}, which means that developers need to be familiar with the underlying technology as well. Thus, Struts has quite a steep learning curve. 

There are many websites built with Struts\footnote{See \url{http://wiki.apache.org/struts/PoweredBy}}, but I didn't find any community platforms. 

\subsubsection{Discussion}
\textbf{Wikis} are good for collaboratively creating simple content. However, they are not intended as a framework for complex applications. PowerPedia is both a web site- and application. Certainly for the application part accessible by mobile phone clients, a framework is needed that supports the development of a RESTful, MVC web application. Following the decision in~\cite{merklepp}, the purpose of wikis does not match the requirements for this project. 
Wikis can be a good supplement for a platform, but this option was not chosen in this case.

Although Energy Monsters seems to have a somewhat similar focus, a mere \textbf{Facebook application} is not suited for PowerPedia. Unlike the Monster App, PowerPedia is primarely a central storage server and only secondary a community platform. PowerPedia's users would have to have a Facebook account, which is not desirable for this project. 

\textbf{Google Web Toolkit} aids front-end application development. It is a client-centric toolkit, but in this project, the server side was more important. 
There are plugins to use GTW as a front-end presentation layer with a web framework such as Ruby on Rails and RESTful web services on the back end\footnote{E.g gwt-rails \url{http://code.google.com/p/gwt-rails/}}, but PowerPedia does not need a rich user interface. Rather, it is a community platform in a Web 2.0 sense. Using GWT for the client-side would be an overkill. 

Unlike dynamic languages, using a Java based framework has the advantage of good development environments with refactoring and other benefits. \textbf{Struts} certainly would have been an option. On the other hand, PowerPedia is no business application. Being able to use the Java EE platform that is intended for large-scale applications would not have been an advantage in this case.  

The decision which framework to used basically boiled down to Grails, Django or Ruby on Rails. 

In the end, I decided to use \textbf{Ruby on Rails}. It was very easy to find information, its documentation is well structured and there are many examples of web platforms that are powered by RoR, especially community ones.
Personally, RoR is my favorite framework, as it makes web application programming fun and easy. Also, testing is tightly integrated and encouraged with the help of tools such as RSpec~\footnote{\url{http://rspec.info/}} and Cucumber~\footnote{\url{http://cukes.info/}}. As described above, one can argue that RoR is not suited for large-scale applications, but this should not be a issue with PowerPedia due to its nature and quite specific use case.  

\textbf{Grails} sounded very interested, but unfortunately, does not seem to be used (yet) in many projects. \textbf{Django} looked very promising as well, one of the big players in web frameworks that has mobilized a big community in the last few years. However, for support for REST, an external plugin has to be used. In contrast, REST is one of the guiding principles in Rails. Apart from this, it would have been a very nice option as well, especially since there seems to be less "magic" behind its scenes compared to Rails, which can be a benefit when it things go wrong.

%Of course, the popularity of a framework can be a deciding factor as well. Its estimation is rather hard, but according to multiple sources\footnote{For example Hotframeworks \url{http://hotframeworks.com/rankings} or Indeed's jobtrends \url{http://www.indeed.com/jobtrends}}, Ruby on Rails is more widespread than both Grails and Django. 

Other options would have been the Java framework Play\footnote{Play \url{http://www.playframework.org/}} or CakePHP\footnote{CakePHP \url{http://cakephp.org}}, a framework for PHP. These framework are all very similar in terms of their functionality and architectural choices. Since I have not worked with any of them, the decision was made in favor of the framework I am already familiar with, especially because of the time constraints of a Master thesis. I am confident though that any of these options including Grails, Django and even Struts would have suited PowerPedia's purpose as well.

\subsection{Interface}
The eMeter system only uses a single sensor, the smart electricity meter, to determine the electricity consumption of a household. In order to provide feedback on a device-level, additional means are necessary. In~\cite{helfensteinsed}, a device recognition algorithm (Smart Energy Detection) was introduced. This algorithm disaggregates the total electricity load into the main consumers based on the measurements retrieved from the smart meter. Measurements made with the mobile phone clients form a knowledge base for the algorithm which should then detect switching events of devices automatically. To be able to do so, the real, reactive, and distortion power sampled at a predetermined frequency is needed. Those values can be inferred or directly acquired from the smart meter (see section TODO).

There are many other appliance load monitoring systems that try to classify and recognize appliances according to certain characteristics. 
%By disaggregating the total load data measured by an electricity meter, individual appliances that are attached to the electrical circuit can be identified. 
Some systems such as~\cite{Farinaccio1999245} apply pattern recognition and only need to know the total power signal in kilowatt hours. Other systems consider changes in voltage as well\cite{hart}. Ruzelli et al.\cite{RuzzelliNSO10} take into account even more values, namely power factor, peak current and sampling rate to support various metering systems. Appliances are then recognized with the help of a neural network, 

Methods used and sensor data needed are manifold. PowerPedia was designed to work with the eMeter system and thus the interface was designed to work with the Smart Energy Detection algorithm. Nonetheless, the possibility to support a wider range of systems in the future can not be excluded. Consequently, the interface was designed flexible enough to cater to other appliance load monitoring and metering systems (compare section TODO).

