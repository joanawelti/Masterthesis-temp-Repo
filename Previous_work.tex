\section{Previous Work} \label{sec:previous_work}
A first version of PowerPedia already exists. It was implemented as part of Adrian Merkle's Master thesis~\cite{merklepp}. This prototype was realized as an extension to the eMeter system, which is described further thereafter.
In order to demonstrate the functionality of Powerpedia, Adrian Merkle also implemented an Android eMeter application based on the already existing iPhone application. This mobile phone application serves as a user interface to access the platform.

\subsection{Overview}
TODO: figure with system overview. Is the source as used in Adrian's thesis available somewhere?

Figure TODO depicts a complete overview of the current status of the system. 
\dots

\subsection{eMeter System}
\begin{itemize}
 \item What is it
 \item What purpose does it serve? Why was it developed?
\end{itemize}

\dots
\subsubsection{eMeter Architecture}\label{sec:emeter_architecture}
The original system was first designed and implemented in a bachelor thesis~\cite{roediger} and consisted of three independent components: a smart electricity meter, a gateway, and a portable user interface on a mobile phone~\cite{weissm:inprocMUM:2009}. 
\dots

\minisec{Smart Electricity Meter}

\minisec{Gateway}

\minisec{Mobile Phone Application}

\subsection{PowerPedia Prototype}
PowerPedia is a community web platform which serves as a central server for consumption measurement data. The prototype was build as an independent module of the eMeter system to complements its features, namely putting electricity consumption values of users in a larger context beyond mere numbers.

PowerPedia was developed as a means to make the energy consumption more transparent for users~\cite{merklepp}. The main objective of the system was to "allow users to better access their electricity consumption and energy efficiency of their appliances"~\cite{weiss:inprocPUC:2012}. To achieve this goal, the PowerPedia prototype is centered around the idea of action guided feedback. It offers the possibility to compare the energy consumption of household appliances to the consumption of devices that reside in the same device category~\cite{merklepp}. User can upload their measurements made with the eMeter system and get immediate feedback. To help users conserve energy, PowerPedia also provides specific tips on how to be more energy efficient.

\subsubsection{Android eMeter Client}
In order to demonstrate the functionality of PowerPedia, a second version of the mobile phone application was developed. This application was realized as an Android application and is based on the one for the iPhone (as described in~\ref{sec:emeter_architecture}). Is used as a user interface to access the platform.

A detailed analysis of the functionality and technical realization of both the web platform and mobile phone application follows in~\ref{sec:prototype_analysis}.