\section{On the quest for a new framework}
- various approaches to evolving a prototype
- reason why prototype is completely rewritten
- rewrite, possibility to reconsider platform choice

Since the old version of PowerPedia was only meant to be a prototype, I considered various possibilities in terms of web application frameworks for the new version. 

\subsection{Framework options}
\subsubsection{Wiki}
Wikis have become quite popular over the last few years. They are used in enterprises to have a central knowledge base as well as by individuals to collect information on a topic in a collaborative way. Wikis are used for a wide variety of topics, such as travel~\footnote{\url{http://wikitravel.org/}} or health~\footnote{\url{http://www.ganfyd.org/}}. The most prominent example is certainly Wikipedia, which has the largest user base~\footnote{WikiStats by S23, S23Wiki, 2008-04-03, retrieved 2007-04-07, \url{http://s23.org/wikistats}} and around 90,000 active contributors.

Because of the popularity of Wikis, users are familiar with the concept. Wikis build upon and encourage collaboration by making the editing and creation of pages easy and available for all users. The structure and format of wiki pages are specified with a simplified markup language in most implementations, although the syntax can vary greatly among them. Especially implementations that don't use "WYSIWYG" ("What You See Is What You Get") editing might be difficult to use for novel users as they first have to become familiar with the specific markup language~\footnote{Wikipedia conducted a usability study in 2009: "Every user in this study struggled to get a basic grasp of the editing interface." \url{http://usability.wikimedia.org/wiki/UX_and_Usability_Study}}. 

\subsubsection{Ruby on Rails}
Ruby on Rails~\footnote{\url{http://rubyonrails.org/}} (RoR) was first introduced in 2003 as a open-source web framework for the Ruby programming language. It implements the Model-View-Controller pattern and empathizes "Convention over Configuration". 

Personally, RoR is my favorite, as it makes web application programming fun and easy. Also, Behavior Driven Development (BDD) is encouraged  with the help of tools such as RSpec~\footnote{\url{http://rspec.info/}} and Cucumber~\footnote{\url{http://cukes.info/}}.
\subsubsection{Recess}